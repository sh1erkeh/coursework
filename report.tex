\documentclass[a4paper,12pt]{article}
\usepackage[utf8]{inputenc}
\usepackage[russian]{babel}
\usepackage{graphicx}
\usepackage{geometry}
\usepackage{array}
\usepackage{longtable}
\usepackage{hyperref}
\usepackage{ulem}
\usepackage{enumitem}

\geometry{left=3cm, right=2cm, top=2cm, bottom=2cm}

\title{Теория графов и хроматические числа в контексте задачи составления расписания}
\author{Андреев Александр Сергеевич, Чапайкин Арсений Георгиевич, группа БИВ242}
\date{Москва 2025}

\begin{document}

\maketitle

\section*{Техническое задание}

\subsection*{1. Формулировка задания}
Разработать программу, которая позволит оптимизировать составление расписания при помощи метода хроматических многочленов.

\subsection*{2. Требования к ПО}
\begin{itemize}
    \item Формирование расписания на основе введённых данных
    \item Автоматическое построение оптимального расписания
    \item Поддержка различных типов расписаний
    \item Ввод и редактирование данных
    \item Управление ограничениями
    \item Оптимизация расписания
    \item Просмотр и визуализация расписания
\end{itemize}

\subsection*{3. Пользовательский интерфейс}
\begin{itemize}
    \item Удобный и интуитивно понятный интерфейс
    \item Графическая визуализация графов
    \item Интерактивные элементы для ввода данных
\end{itemize}

\section*{Введение}
Теория графов является одной из наиболее динамично развивающихся областей математики и информатики...

\section*{Обзор аналогов}
\begin{itemize}
    \item Google Calendar
    \item Microsoft Outlook
    \item Trello
    \item Asana
\end{itemize}

\section*{Описание выбора программных средств}
Для разработки выбраны:
\begin{itemize}
    \item Язык программирования C++
    \item Qt Framework (версия 6)
    \item Система сборки CMake
    \item Среда разработки Qt Creator
\end{itemize}

\section*{Описание алгоритма}
\subsection*{Алгоритм DSatur}
\begin{enumerate}
    \item Инициализация:
    \begin{itemize}
        \item Для каждой вершины $v \in V$:
        \item $color(v) = NULL$
        \item $sat(v) = 0$
        \item Вычислить $deg(v)$
    \end{itemize}
    
    \item Основной цикл:
    \begin{enumerate}
        \item Выбор вершины с максимальной насыщенностью
        \item Присвоение минимально возможного цвета
        \item Обновление насыщенности соседей
    \end{enumerate}
\end{enumerate}

\section*{Руководство пользователя}
\begin{figure}[h]
\centering
\includegraphics[width=0.8\textwidth]{interface.png}
\caption{Главный экран приложения}
\end{figure}

\section*{Заключение}
В ходе выполнения курсовой работы была разработана программа для решения задачи составления расписания...

\begin{thebibliography}{9}
\bibitem{wiki} Теория расписания // Википедия. URL: \url{https://ru.wikipedia.org/wiki/Теория_расписаний}
\bibitem{habr} Алгоритмы раскраски графов // Habr. URL: \url{https://habr.com/ru/articles/467005/}
\end{thebibliography}

\end{document}
